\section{Question 3}

\subsection{Task}
Consider a temperature sensor placed near a valuable asset in a highly secure space. It records the local temperature every 5 seconds and writes it to a stream. The temperature data is flushed such that it keeps the data of only one hour. That is, you can view a window of only 720 data points at a time. Each new data point entering the window flushes the oldest data point out. Write two programs that do the following tasks.
\begin{itemize}
	\item Program-A when executed will continuously send temperature data that looks almost flat (say, room temperature with a small random noise within 0.5 Kelvin). This is to simulate the data coming from the temperature sensor. At predetermined instance, introduce a spike (sudden increase of temperature by 10 K) for 10 seconds and revert back to room temperature. This spike is caused due to the presence of an intruder.
	\item Program-B keeps reading the incoming data, detects the spike and reports the instance when it took place. Verify if the detection is accurate.
\end{itemize}
Think of the output from Program-A being piped to Program-B to perform this check. The stderr from both the programs about the spike should match. The stdout of Program-A contains the temperature data. You should submit a report on how this is made along with the two codes.

\subsection{Solution}

Link to the GitHub repository for this question: \href{https://github.com/Xerefic/MM2090-Solutions/tree/master/Final_Assignment/question_3}{GitHub} \footnote{Repo: \url{https://github.com/Xerefic/MM2090-Solutions/tree/master/Final_Assignment/question_3}}